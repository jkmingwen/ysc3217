\documentclass[parallelismlabreport.tex]{subfiles}

\begin{document}

\section{Sequential implementation}
\subsection{checkArray\_seq function}
The first thing I did was to implement a function that would take in an array of digits; it would ensure that the digits were between 1 and 9, and then check if there were any repeated digits in the given array. The code for this function follows:
\begin{lstlisting}[language=C]
int checkArray_seq(int n[])
{
    int counter = 0;

    for (int j = 0; j < 9; j++) // this acts as our checker array
    {
	for (int i = 0; i < 9; i++)
	{
	    if (n[i] == j + 1)
	    {
		counter++;
		break;
	    }
	}
    }
    if (counter != 9)
    {
	return 1;
    }
}
\end{lstlisting}
Each element of the input array, \verb|n[i]| is compared with the variable \verb|j + 1| (this starts with 1). The moment an element matches, the \verb|counter| is incremented by 1 and the \verb|break| function is called. The elements of the input array are then compared to the subsequent digit (in this case, 2). As \verb|j + 1| is only matched to its first match, a valid sudoku solution should result in a final \verb|counter| value of 9. If it is any value other than 9, the solution is invalid and the function returns 1.
\subsection{checkSudoku\_seq function}
The \verb|checkSudoku_seq| function takes in a pointer to a 2 dimensional array flattened in memory to a 1 dimensional array. It then checks --- in sequence --- the rows, columns and 3x3 grids for validity as a sudoku solution using the \verb|checkArray_seq| function. It returns its result to \verb|sudoku_fail| --- an array of 27 elements. Finally, using the OR boolean operator, the array of results are checked for any instances of 1s. If, after running the check, all elements of \verb|sudoku_fail| are 0, the solution is declared valid. Otherwise, the solution is declared invalid.
\subsection{Difficulties faced}
The two challenges faced in writing this solution were (1) figuring out a way to check for repeating digits, and (2) checking the 3x3 grids by traversing a one dimensional array. The solution might not be the most elegant --- especially the way it traverses through the 3x3 grids --- but it does work for the tests implemented.
\end{document}